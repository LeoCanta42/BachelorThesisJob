\chapter{Future Works} 

\section{Fuzzing with Reinforcement Learning}
Although Fallaway has shown promising results in our experiments, there is still room for improvement. One of the most viable approaches to achieving an optimal configuration involves the use of Reinforcement Learning (RL).
\\Reinforcement Learning is a subfield of machine learning in which an agent learns to make decisions through interaction with an environment. During this interaction, the agent performs a series of actions and receives rewards or penalties based on those actions. The agent uses this feedback to learn an optimal policy — a strategy for selecting actions that maximizes the cumulative reward over time. A key concept in RL is the trade-off between exploration (trying new actions to discover their effects) and exploitation (choosing known actions that yield high rewards).
\\By applying RL to fuzzing, the system can automatically adapt parameters such as mutation rates, seed selection strategies, and other relevant configurations to maximize code coverage. In this context, the RL agent learns to identify and prioritize inputs that are likely to trigger new edge cases or explore untested paths in the code. Each time the fuzzer discovers a new edge case or covers a previously unexplored branch, it receives a reward. Over time, the agent refines its strategy to generate inputs that are more effective at increasing coverage, thereby enabling a more comprehensive exploration of the software under test.
\\Integrating RL into the configuration process of fuzzers allows for continuous optimization of the fuzzer’s behavior at runtime. This dynamic adaptation can significantly enhance the fuzzer's ability to find vulnerabilities and other critical software defects.

\section{Fuzzing with multi-process SUTs}
Another one interesting future avenue of work is to extend Fallaway for fuzzing multi-process SUTs.
\\There are already some works on this field, let me cite some: \textit{Evaluating the Fork-Awareness of Coverage-Guided Fuzzers} \cite{icissp23}, \textit{Forkfuzz: Leveraging the Fork-Awareness in Coverage-Guided Fuzzing} \cite{forkfuzz}.
\\In many modern software systems, most of which are designed to provide high performance and scalability, different tasks or their stages of execution in most cases are performed by multiple processes in cooperation. Web servers (like even Lighttpd) are an example where a main process would listen constantly for incoming network connections while spawning child processes with every request handled separately. That architectural pattern allows improvement in both responsiveness and fault isolation; it also presents new challenges when performing fuzzing.
\\Fuzzing multi-process systems typically relies on crafting inputs that effectively exercise the communication and interactions between the different processes. This becomes more complex than pure single-process fuzzing, as the fuzzer has to orchestrate the execution of multiple processes together: it needs to make sure that inputs are correctly synchronized and passed between the various processes running in parallel or at a different stage of execution.
\\To this end, the fuzzer would need to understand not only the data that flows between the processes involved but also the timing and state dependencies inherent in such a setup. In other words, it needs to generate inputs that trigger specific sequences of interactions between the main process and its spawned child processes or between processes that talk to each other through shared memory, sockets, or any of the other IPC mechanisms available.
\\For Fallaway to handle such cases, further extension would be needed for increased multi-process tracking and management of the execution flow together with monitoring and manipulating effectively the IPC channels. It might include developing hooking techniques into the paths of communications, capturing and mutating the messages which processes exchange or introducing small delays under control in order to explore different interleavings of executions. Overcoming this would allow Fallaway to fuzz modern software systems more comprehensively, which could be exposing bugs that only appear under multiprocessor environments.