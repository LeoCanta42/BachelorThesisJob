\chapter{Conclusion} 
\label{chap:Conclusion}
This thesis explored the effectiveness of three stateful fuzzers—Fallaway, AFLNet, and ChatAFL—when applied to the Lighttpd web server. The study aimed to evaluate each tool, focusing on aspects like coverage, efficiency, and adaptability.
\\\\The importance of stateful fuzzing has been discussed, particularly for applications where internal states and state transitions significantly impact behavior, such as web servers and network-based applications. Stateful fuzzing techniques were shown to be crucial in effectively testing these SUTs, as they consider the influence of previous interactions on the application's current state.
\\\\The research also involved comparing the setup processes for each fuzzer. Each tool required a different configuration and environment setup to achieve optimal performance. For instance, Fallaway's modifications to Lighttpd required a persistent mode to maintain server states across multiple requests. ChatAFL leverages large language models (LLMs) for generating inputs, while AFLNet utilizes a network-aware approach to enhance its fuzzing capabilities.
\\\\Additionally, various graphs were presented to illustrate the comparative results of the fuzzers. These showed differences in execution counts, code coverage achieved over time, and other performance metrics, providing a clearer understanding of how each tool behaves in different scenarios. 
\\\\\textbf{Fallaway} was observed to extensively explore possible program behaviors, achieving broad coverage by conducting numerous test executions. Its method is advantageous in scenarios where a comprehensive examination of all potential states is crucial.
\\\\On the other hand, \textbf{AFLNet} leveraged its specialization in network protocols to achieve meaningful results with fewer test cases. It effectively targeted specific parts of the code, making it suitable for applications that require testing of network-related functionalities. However, its narrower focus might limit its applicability to broader testing needs.
\\\\\textbf{ChatAFL} employs an innovative approach by utilizing advanced techniques to enhance input generation. By leveraging large language models (LLMs) to generate inputs and effectively overcoming coverage plateaus, it offers a strong solution for fuzzing. However, it remains somewhat less effective compared to Fallaway.
\\\\The findings show that Fallaway has achieved more results, despite its relatively simpler approach. The persistent mode significantly contributes to its effectiveness and the knowledge of the state model in Lighttpd, even if it is somewhat limited due to the absence of real proper well-defined states, still enables Fallaway to produce better results.

\section{Future Works}
Future work could explore combining the efficiency of persistent mode with approaches based on large language models (LLMs).
\\Another potential area of investigation is fuzzing Lighttpd in a multiprocess setting, without limiting the number of worker processes to one. This would require developing or employing fork-aware fuzzing techniques ~\cite{forkfuzz,icissp23}, which is left as future research.
