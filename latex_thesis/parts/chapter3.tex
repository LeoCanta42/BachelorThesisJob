\chapter{Setup and fuzzing}
\label{chap:Setup}

\section{Fallaway}
Fallaway distinguishes itself from other fuzzers like AFLNet or ChatAFL by using a \textbf{persistent mode}. This mode allows the fuzzer to maintain the server's state across multiple requests, which is especially useful in scenarios where the server does not reset its state between requests, such as when managing user sessions or maintaining authentication states in a web application.
\\The persistent mode is implemented by modifying the Lighttpd server to maintain its state between requests. The server operates in a separate process and the fuzzer interacts with it via a socket. The fuzzer sends requests to the server and receives responses, using the results to guide the generation of subsequent requests. This process continues in a loop until the fuzzing session is complete.
\\To enable this, it is necessary to modify the Lighttpd code to ensure that the server continuously receives, processes and responds to requests without shutting down. The changes are made to the function \textit{server\_main\_loop} in the \textit{src/server.c} file and to the connection handling functions in \textit{src/connections.c} of the Lighttpd source code. The specific code changes are shown in the next section, providing a comparison between the original and modified code.

\subsection{Lighttpd Code Modifications for Persistent Mode}

Table \ref{tab:connections_patch} presents a comparison of the original and modified code for the \textit{connections.c} file. The modifications to this file are crucial for maintaining an open connection state, ensuring that the fuzzer can interact continuously with the server. It is also important to clean all buffers and old data for that connection.
\\Table \ref{tab:server_patch} shows a comparison of the original and modified code for the \textit{server.c} file. The changes made here are essential for enabling persistent server operation, allowing the fuzzer to manage and maintain server state across multiple requests, looping into the \textit{\_\_AFL\_LOOP}.

\begin{table}[H]
\centering
\begin{tabular}{|p{0.9\textwidth}|}
\hline
\textbf{Original Code} \\
\hline
\begin{lstlisting}[language=c, basicstyle=\footnotesize, breaklines=true]
static void connection_handle_shutdown(connection *con) {
    ...
    connection_reset(con);
    
    /* close the connection */
    if (con->fd >= 0 
        && (con->is_ssl_sock 
        || 0 == shutdown(con->fd, SHUT_WR))) {
        con->close_timeout_ts = log_monotonic_secs;

        request_st * const r = &con->request;
        connection_set_state(r, CON_STATE_CLOSE);
        if (r->conf.log_state_handling) {
            log_error(r->conf.errh, __FILE__, __LINE__,
                "shutdown for fd %d", con->fd);
        }
    } else {
        connection_close(con);
    }
}
\end{lstlisting} \\
\hline
\textbf{Modified Code} \\
\hline
\begin{lstlisting}[language=c, basicstyle=\footnotesize, breaklines=true]
static void connection_handle_shutdown(connection *con) {
    ...
    connection_reset(con);

    /* keep the connection open and reset it */
    request_reset_ex(&con->request);
    chunkqueue_reset(con->read_queue);
    con->request_count = 0;
    con->is_ssl_sock = 0;
    con->revents_err = 0;
    connection_set_state(&con->request,CON_STATE_REQUEST_START);
}
\end{lstlisting} \\
\hline
\end{tabular}
\caption{Comparison of Original and Modified Code for `src/connections.c`}
\label{tab:connections_patch}
\end{table}

\begin{table}[H]
\centering
\begin{tabular}{|p{0.9\textwidth}|}
\hline
\textbf{Original Code} \\
\hline
\begin{lstlisting}[language=c, basicstyle=\footnotesize, breaklines=true]
static void server_main_loop (server * const srv) {
    ...
    server_load_check(srv);

    #ifndef _MSC_VER
    static
    #endif
    connection * const joblist = log_con_jqueue;
    log_con_jqueue = sentinel;
    server_run_con_queue(joblist, sentinel);

    if (fdevent_poll(srv->ev, log_con_jqueue != sentinel ? 0 : 1000) > 0)
        last_active_ts = log_monotonic_secs;
}
\end{lstlisting} \\
\hline
\textbf{Modified Code} \\
\hline
\begin{lstlisting}[language=c, basicstyle=\footnotesize, breaklines=true]
static void server_main_loop (server * const srv) {
    ...
    server_load_check(srv);

    while (__AFL_LOOP(INT64_MAX)) {
        fdevent_poll(srv->ev, -1);
        
        #ifndef _MSC_VER
        static
        #endif
        connection * const joblist = log_con_jqueue;
        log_con_jqueue = sentinel;
        server_run_con_queue(joblist, sentinel);
    }

    srv_shutdown = 1;
}
\end{lstlisting} \\
\hline
\end{tabular}
\caption{Comparison of Original and Modified Code for `src/server.c`}
\label{tab:server_patch}
\end{table}

\subsection{Mutator and Corpus}

Another crucial aspect of the fuzzing process involves the corpus and the mutator. In this experiment, it has been defined the state of the server based on the existence \ref{tab:existent_resource} or non-existence \ref{tab:nonexistent_resource} of a resource. Specifically considering two types of requests: one that attempts to access a resource that exists and another that attempts to access a resource that does not exist.
\\\\The \textbf{corpus} folder consists of: a set of folders, one for each state, each containing a set of files, the \textit{prefixes}, that are a sequence of messages to reach that state.
\\In this case, the corpus is composed by two folders: one for the existent resource and one for the non-existent resource. Each folder contains a single file with a request to reach the state.
Here we have two requests for the two states:
\begin{figure}[H]
    \centering
    \begin{adjustbox}{valign=t}
    \begin{lstlisting}
    PUT /hello.txt HTTP/1.1
    Host: 127.0.0.1:8080
    Content-type: text/plain
    Content-length: 13

    Hello, World!

    
    \end{lstlisting}
    \end{adjustbox}
    \caption{Existent resource request}
    \label{tab:existent_resource}
\end{figure}
    
\begin{figure}[H]
    \centering
    \begin{adjustbox}{valign=t}
    \begin{lstlisting}
    DELETE /hello.txt HTTP/1.1
    Host: 127.0.0.1:8080
    User-Agent: curl/8.0.1
    Accept: */*

    
    \end{lstlisting}
    \end{adjustbox}
    \caption{Non-existent resource request}
    \label{tab:nonexistent_resource}
\end{figure}
\phantom{}\\
The \textbf{mutator}, responsible for generating variations of the requests, is relatively straightforward. Its primary function is to modify the existing requests by appending the sequence of characters ``\textit{\textbackslash r\textbackslash n\textbackslash r\textbackslash n}" to the end of each request. This modification is essential as it ensures that the requests are well-formed and adhere to HTTP protocol standards.
\\Before this, the mutator adds a set of tokens (taken from files in the corpus folder, but outside of the state folder) and places them in random positions within the request.
\\By ensuring the requests are properly formatted, the mutator enables the server to parse and process them correctly, which is vital for accurate fuzz testing.
\\An example of the corpus folder is as follows:
\begin{figure}[H]
    \centering
    \begin{adjustbox}{valign=t}
    \begin{lstlisting}
    corpus
    |-- existent_resource
    |   |-- 0_put
    |   |-- metadata
    |-- non_existent_resource
    |   |-- 0_delete
    |   |-- metadata
    |-- GET
    |-- DELETE
    |-- PUT
    |-- OPTIONS
    |-- POST
    \end{lstlisting}
    \end{adjustbox}
    \caption{Corpus folder structure}
\end{figure}
\phantom{}\\
Summing up: \textit{existent\_resource} folder is a state, \textit{0\_put} is a prefix to reach that state and, in this case is the full request to reach it. The same for \textit{non\_existent\_resource} and \textit{0\_delete}. The \textit{metadata} files contains the number of outgoing edges from that state.
\\Finally the other files rapresent the tokens that the mutator will use to generate new requests.

\subsection{Setting Up the Fuzzing Environment}

To run the fuzzer, it is important to build a Docker container that includes all the necessary dependencies and the modified Lighttpd server. The Dockerfile below is based on an image that already contains Fallaway and shows the steps to set up this environment.

\begin{lstlisting}
FROM fallaway 

WORKDIR /

# Copy the patch file
COPY ./lighttpd.patch /lighttpd.patch

ENV DEBIAN_FRONTEND=noninteractive

# Install lighttpd dependencies
RUN apt-get install -y \
    autoconf \
    automake \
    libtool \
    m4 \
    pkg-config \
    libpcre2-dev \
    zlib1g-dev \
    zlib1g \
    openssl \
    libssl-dev \
    scons

# Create the root directory for the server
RUN chmod 777 /tmp

# Install

# Set up environment variables for ASAN
ENV ASAN_OPTIONS='abort_on_error=1:symbolize=0:detect_leaks=0:detect_stack_use_after_return=1:detect_container_overflow=0:poison_array_cookie=0:malloc_fill_byte=0:max_malloc_fill_size=16777216'

# Download lighttpd
ENV CC=afl-cc
ENV CXX=afl-cc
RUN git clone https://git.lighttpd.net/lighttpd/lighttpd1.4.git lighttpd
WORKDIR /lighttpd 
RUN git checkout 9f38b63cae3e2
RUN git apply /lighttpd.patch
RUN ./autogen.sh  
RUN scons CC=/AFLplusplus/afl-cc CXX=/AFLplusplus/afl-cc -j 4 build_static=1 build_dynamic=0
RUN mv /lighttpd/sconsbuild/static/build/lighttpd /lighttpd/lighttpd

# Copy the corpus
COPY ./corpus /corpus

# Copy the config file
COPY ./lighttpd.conf /lighttpd.conf

# Copy the run script
COPY ./run.sh /Fallaway/run.sh
# Make it executable
RUN chmod +x /Fallaway/run.sh

WORKDIR /Fallaway
\end{lstlisting}
\phantom{}\\
The Docker container is configured with all the dependencies to run the fuzzer and build the modified Lighttpd server, providing a controlled environment to conduct the fuzzing experiment.
Another important file to consider is the configuration file of the Lighttpd server, which is shown below.\\
\begin{lstlisting}
    server.document-root = "/tmp"
    server.bind = "0.0.0.0"
    server.port = 8080
    mimetype.assign = (".txt" => "text/plain", ".html" => "text/html" )

    server.max-worker = 1 
    server.max-connections = 1000
\end{lstlisting}
\phantom{}\\
This configuration file specifies the server's document root, bind address, port, and maximum number of workers and connections. By defining these parameters the server will operates correctly and can handle the incoming requests from the fuzzer.
\\In particular, it is forced to have just one worker to avoid problems with fuzzing, because Fallaway is not designed to work with multi-process SUT.
\subsection{Fuzzing Execution and Results}

To run the fuzzer for 24 hours, it is necessary to run the following script:
\begin{lstlisting}
#!/bin/bash
bin="${1:-mcsm-cy}"
loops="${2:-1000}"

timeout 24h cargo run --release --bin fallaway-http-$bin -- --in-dir /corpus --out-dir /output_lighttpd --target-port 8080 --loops $loops -t 300 /lighttpd/lighttpd -D -f /lighttpd.conf
\end{lstlisting}
In particular there are:
\begin{itemize}
    \item \textbf{timeout 24h}: a timeout of 24h for the next command.
    \item \textbf{cargo run}: the command to run the fuzzer.
    \item \textbf{--release}: the flag to run the fuzzer in release mode.
    \item \textbf{--bin fallaway-http-\$bin}: the state scheduler strategy (by default is mcsm-cy).
    \item \textbf{--}: the flag to separate the fuzzer arguments from the binary arguments.
    \item \textbf{--in-dir /corpus}: the input directory for the fuzzer.
    \item \textbf{--out-dir /output\_lighttpd}: the output directory for the fuzzer, where the results will be stored.
    \item \textbf{--target-port 8080}: the port of the server.
    \item \textbf{--loops \$loops}: the number execution the fuzzer will do before changing state (the \textit{\_\_AFL\_LOOP} is bigger volountarly, so that we prioritize this argument).
    \item \textbf{-t 300}: the timeout for each test case, in milliseconds, which will trigger if the fuzzer does not reach the end of the \textit{\_\_AFL\_LOOP} in time.
    \item \textbf{/lighttpd/lighttpd -D -f /lighttpd.conf}: the command to run the server, in detached mode, with the configuration file.
\end{itemize}
The results of the fuzzing process will be discussed in the Chapter \ref{chap:Results}.

\section{Comparison with AFLNet and ChatAFL}

AFLNet and ChatAFL provide alternative approaches to fuzzing that share certain characteristics with Fallaway, but also have distinct differences in their setup, configuration, and operational strategies. The purpose of this section is to outline the setup for both AFLNet and ChatAFL, noting similarities with Fallaway, and highlight the unique features of each tool, including how they handle server responses and their internal mechanisms for optimizing fuzzing performance.

\subsection{Setting up the fuzzing environment}

The setup process for AFLNet and ChatAFL is quite similar to that of Fallaway, given that all three fuzzers share a common Docker-based environment with the necessary dependencies. Both AFLNet and ChatAFL are built and configured using \textbf{ProFuzzBench} ~\cite{profuzzbench}, a benchmark suite specifically designed for evaluating network protocol fuzzers. ProFuzzBench provides a standardized environment and set of targets to ensure a fair comparison among different fuzzers.
\\By using ProFuzzBench, AFLNet and ChatAFL benefit from a streamlined setup process that automates the installation of dependencies and configuration of the environment, thus reducing setup overhead. This setup also involves additional dependencies, such as specific Python packages, which are necessary for supporting ChatAFL's unique capabilities like leveraging language models internally.
\\The Docker setup derived from ProFuzzBench provides the same base environment for both AFLNet and ChatAFL, ensuring compatibility and consistency across experiments. By using this common benchmark suite, researchers can directly compare results, further validating the effectiveness and performance differences between the fuzzers.
Another important thing to consider is the configuration file of the Lighttpd server, which is shown below.\\
\begin{lstlisting}
    server.document-root = "/tmp"
    server.bind = "127.0.0.1"
    server.port = 8080
    mimetype.assign = (".txt" => "text/plain", ".html" => "text/html" )
\end{lstlisting}
\phantom{}\\
As seen above, in this case there is no need to force the number of workers to 1, because they do not have the same problem as Fallaway, due to their managment of the SUT, explained in the next section.

\subsection{Mutator and Corpus Management}

Similar to Fallaway, AFLNet and ChatAFL use a corpus of test cases to seed the fuzzing process. However, their approach to handling and mutating this corpus differs slightly:

\begin{itemize}
    \item \textbf{AFLNet}: Focuses on network protocol fuzzing by analyzing and mutating protocol-specific fields in input messages. The corpus for AFLNet includes various protocol messages (e.g., HTTP requests) that are tailored to network targets. AFLNet leverages coverage feedback as well as response error codes from the server to refine its mutations and generate new test cases.

    \item \textbf{ChatAFL}: Enhances the mutation process using a language model (LLM) to generate intelligent mutations. This approach allows it to craft inputs that are more likely to uncover new code paths or trigger complex behaviors. The LLM is used to predict and prioritize inputs based on semantic understanding of the protocol or application under test.
\end{itemize}

Here are some examples of the corpus used by AFLNet and ChatAFL:
\begin{figure}[H]
    \centering
    \begin{adjustbox}{valign=t}
    \begin{lstlisting}
    GET /hello.txt HTTP/1.1
    Host: 127.0.0.1:8080
    User-Agent: curl/8.0.1
    Accept: */*
        
            
    \end{lstlisting}
    \end{adjustbox}
    \caption{Seed used by AFLNet and ChatAFL}
\end{figure}

\begin{figure}[H]
    \centering
    \begin{adjustbox}{valign=t}
    \begin{lstlisting}
    OPTIONS /hello.txt HTTP/1.1
    Host: 127.0.0.1:8080
    User-Agent: curl/8.0.1
    Accept: */*

    
    \end{lstlisting}
    \end{adjustbox}
    \caption{Seed used by AFLNet and ChatAFL}
\end{figure}

\begin{figure}[H]
    \centering
    \begin{adjustbox}{valign=t}
    \begin{lstlisting}
    DELETE /hello.txt HTTP/1.1
    Host: 127.0.0.1:8080
    User-Agent: curl/8.0.1
    Accept: */*

    
    \end{lstlisting}
    \end{adjustbox}
    \caption{Seed used by AFLNet and ChatAFL}
\end{figure}
\phantom{}\\
AFLNet and ChatAFL also use a dictionary during fuzzing, shown in Figure \ref{fig:dictionary}. This dictionary is used to generate meaningful and diverse input cases, ensuring that the fuzzer explores a wide range of scenarios and protocols. By leveraging such a dictionary, these tools enhance their ability to cover different code paths.
\begin{figure}[H]
    \centering
    \begin{adjustbox}{valign=t}
    \begin{lstlisting}
    "GET"
    "PUT"
    "POST"
    "OPTIONS"
    "127.0.0.1"
    "DELETE"
    "CONNECT"
    "TRACE"
    "HEAD"
    "hello.txt"
    "User-Agent"
    "StarWars3.wav"
    \end{lstlisting}
    \end{adjustbox}
    \caption{Dictionary used by AFLNet and ChatAFL}
    \label{fig:dictionary}
\end{figure}

\subsection{Operational Differences: Code Handling and Analysis}

One of the main distinctions between Fallaway, AFLNet, and ChatAFL is their strategy for guiding the fuzzing process:

\begin{itemize}
    \item \textbf{AFLNet} and \textbf{ChatAFL}: Both tools incorporate error code analysis in their feedback loop. They monitor the response codes (such as HTTP 404, 500, etc.) returned by the server to understand which inputs trigger errors or unexpected states. This allows them to focus on generating inputs that might exploit these observed errors, thereby uncovering potential vulnerabilities.
    \item \textbf{Fallaway}: In contrast, Fallaway exclusively relies on coverage metrics to guide the fuzzing process. It focuses on maximizing the code paths exercised by the generated inputs without directly considering the response codes from the server. This approach enables it to explore new paths more thoroughly, but may overlook specific error states that are of interest for security testing.
\end{itemize}

\subsection{Fuzzing Execution and Results}

The real difference from Fallaway is in the execution script:\\
\begin{lstlisting}
    #!/bin/bash

    PFBENCH="$PWD/benchmark"
    cd $PFBENCH

    PATH=$PATH:$PFBENCH/scripts/execution:$PFBENCH/scripts/analysis
    NUM_CONTAINERS=$1
    TIMEOUT=$(( ${2:-1440} * 60))
    SKIPCOUNT="${SKIPCOUNT:-1}"
    TEST_TIMEOUT="${TEST_TIMEOUT:-5000}"

    export TARGET_LIST=$3
    export FUZZER_LIST=$4

    if [[ "x$NUM_CONTAINERS" == "x" ]] || [[ "x$TIMEOUT" == "x" ]] || [[ "x$TARGET_LIST" == "x" ]] || [[ "x$FUZZER_LIST" == "x" ]]
    then
        echo "Usage: $0 NUM_CONTAINERS TIMEOUT TARGET FUZZER"
        exit 1
    fi

    PFBENCH=$PFBENCH PATH=$PATH NUM_CONTAINERS=$NUM_CONTAINERS TIMEOUT=$TIMEOUT SKIPCOUNT=$SKIPCOUNT TEST_TIMEOUT=$TEST_TIMEOUT scripts/execution/profuzzbench_exec_all.sh ${TARGET_LIST} ${FUZZER_LIST}
\end{lstlisting}
\phantom{}\\
An example of execution line is like this:\\
\begin{lstlisting}
./run.sh  <container_number> <fuzzed_time> <subjects> <fuzzers>
\end{lstlisting}
\phantom{}\\
The script takes four arguments:\\
\begin{itemize}
    \item \textit{CONTAINER\_NUMBER}: the number of containers to use for the execution of the fuzzer.
    \item \textit{FUZZED\_TIME}: the time in minutes after which the execution of the fuzzer will be stopped.
    \item \textit{SUBJECTS}: a list of targets to fuzz.
    \item \textit{FUZZERS}: a list of fuzzers to use.
\end{itemize}
\phantom{}\\
The script to fuzz Lighttpd for 24h using both AFLNet and ChatAFL is:
\begin{lstlisting}
./run.sh 1 1440 lighttpd aflnet,chatafl
\end{lstlisting}
Results from both AFLNet and ChatAFL will be discussed in detail in Chapter \ref{chap:Results}.